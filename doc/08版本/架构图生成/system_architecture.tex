
\documentclass[border=8pt, multi, tikz]{standalone}
\usepackage{tikz}
\usepackage{xcolor}
\usepackage{amsmath}
\usepackage{xeCJK}
\setCJKmainfont{Noto Sans CJK SC}

\usetikzlibrary{positioning, arrows.meta, shapes.geometric, calc, fit, backgrounds}

% 定义颜色
\definecolor{layer1}{RGB}{230,230,250}  % 数据层 - 淡紫色
\definecolor{layer2}{RGB}{255,228,181}  % 特征层 - 淡橙色
\definecolor{layer3}{RGB}{152,251,152}  % 推理层 - 淡绿色
\definecolor{layer4}{RGB}{135,206,250}  % 应用层 - 淡蓝色

\begin{document}
\begin{tikzpicture}[
    node distance=0.8cm and 1.5cm,
    layer/.style={rectangle, draw, thick, minimum width=14cm, minimum height=2.5cm, align=center, font=\small},
    module/.style={rectangle, draw, minimum width=3.5cm, minimum height=1.2cm, align=center, font=\footnotesize, rounded corners=3pt},
    arrow/.style={-Stealth, very thick},
    label/.style={font=\scriptsize, align=center}
]

% ==================== 第1层:数据管理层 ====================
\node[layer, fill=layer1!40] (L1) at (0,0) {};
\node[above=0.1cm of L1.north, font=\bfseries\large] {Layer 1: 数据管理层};

\node[module, fill=layer1!60] (mysql) at (-4, 0) {MySQL 8.0\\元数据库};
\node[module, fill=layer1!60, right=0.5cm of mysql] (redis) {Redis 7.0\\特征缓存};
\node[module, fill=layer1!60, right=0.5cm of redis] (minio) {MinIO\\视频存储};

\node[label, below=0.3cm of redis, align=center] {存储:视频45min≈450MB (H.265) | 特征70维×270片段≈2MB | TTL=7天};

% ==================== 第2层:特征提取层 ====================
\node[layer, fill=layer2!40, above=3cm of L1] (L2) {};
\node[above=0.1cm of L2.north, font=\bfseries\large] {Layer 2: 特征提取层 (Pipeline并行)};

\node[module, fill=layer2!70] (video_pipe) at (-4.2, 4.5) {
    \textbf{视频流水线}\\
    YOLOv8 → DeepSORT\\
    MediaPipe → ST-GCN\\
    \textcolor{red}{$F_v \in \mathbb{R}^{20}$}
};

\node[module, fill=layer2!70, right=0.3cm of video_pipe] (audio_pipe) {
    \textbf{音频流水线}\\
    Whisper → Wav2Vec2\\
    情感识别\\
    \textcolor{red}{$F_a \in \mathbb{R}^{15}$}
};

\node[module, fill=layer2!70, right=0.3cm of audio_pipe] (text_pipe) {
    \textbf{文本流水线}\\
    BERT → H-DAR\\
    NLP统计\\
    \textcolor{red}{$F_t \in \mathbb{R}^{35}$}
};

\node[label, below=0.3cm of audio_pipe, align=center] {
    耗时:视频0.82s | 音频0.37s | 文本0.15s | \textbf{总计0.82s/10s片段}
};

% 连接Layer 1 -> Layer 2
\draw[arrow] (minio.north) -- ++(0, 0.8) -| (video_pipe.south);
\draw[arrow] (minio.north) -- ++(0, 0.8) -| (audio_pipe.south);
\draw[arrow] (redis.north) -- ++(0, 0.8) -| (text_pipe.south);

% ==================== 第3层:模型推理层 ====================
\node[layer, fill=layer3!40, above=3cm of L2] (L3) {};
\node[above=0.1cm of L3.north, font=\bfseries\large] {Layer 3: 模型推理层};

\node[module, fill=layer3!70, minimum width=5cm, minimum height=1.5cm] (shape) at (-2.5, 9) {
    \textbf{SHAPE 融合模型}\\
    跨模态注意力 + BiLSTM\\
    342K参数 | 推理0.016s\\
    \textcolor{blue}{7类风格分类 (93.5\%)}
};

\node[module, fill=layer3!70, minimum width=5cm, minimum height=1.5cm, right=0.5cm of shape] (shap) {
    \textbf{SHAP 解释器}\\
    特征归因分析\\
    64背景样本\\
    \textcolor{blue}{可解释性分析}
};

\node[label, below=0.3cm of shape, align=center] {
    GPU加速 (TensorRT) | 批处理10x加速 | 注意力权重可视化
};

% 连接Layer 2 -> Layer 3
\draw[arrow] (video_pipe.north) -- ++(0, 0.8) -| (shape.south);
\draw[arrow] (audio_pipe.north) -- ++(0, 0.5) -- (shape.south);
\draw[arrow] (text_pipe.north) -- ++(0, 0.8) -| (shape.south);

% ==================== 第4层:应用服务层 ====================
\node[layer, fill=layer4!40, above=3cm of L3] (L4) {};
\node[above=0.1cm of L4.north, font=\bfseries\large] {Layer 4: 应用服务层};

\node[module, fill=layer4!70] (profile) at (-4.2, 13.5) {
    \textbf{画像生成}\\
    风格雷达图\\
    典型片段提取
};

\node[module, fill=layer4!70, right=0.3cm of profile] (viz) {
    \textbf{可视化}\\
    行为柱状图\\
    情绪曲线
};

\node[module, fill=layer4!70, right=0.3cm of viz] (analysis) {
    \textbf{分析服务}\\
    SMI相似度\\
    稳定性追踪
};

% 连接Layer 3 -> Layer 4
\draw[arrow] (shape.north) -- ++(0, 0.8) -| (profile.south);
\draw[arrow] (shape.north) -- ++(0, 0.5) -- (viz.south);
\draw[arrow] (shap.north) -- ++(0, 0.8) -| (analysis.south);

% ==================== 用户接口 ====================
\node[module, fill=layer4!90, above=1cm of viz, minimum width=8cm, minimum height=1cm] (ui) {
    \textbf{用户接口} (Vue.js + ECharts)\\
    教师端:风格画像查看 | 教研端:批量分析、跨教师对比
};

\draw[arrow] (profile.north) -- ++(0, 0.3) -| (ui.south);
\draw[arrow] (viz.north) -- (ui.south);
\draw[arrow] (analysis.north) -- ++(0, 0.3) -| (ui.south);

% ==================== 关键设计标注 ====================
\node[label, below=2.2cm of L1.south, draw, dashed, thick, minimum width=14cm, minimum height=2cm, align=left] (design) {
    \textbf{关键设计:}\\
    1. \textbf{异步任务队列} (Celery + RabbitMQ):支持批量处理,失败重试\\
    2. \textbf{三级缓存策略}:L1=模型输出(Redis, 24h) | L2=特征向量(Redis, 7d) | L3=视频文件(MinIO, 永久)\\
    3. \textbf{水平扩展}:特征提取服务与模型推理服务可独立扩容,支持50并发/单机 → 200并发/分布式
};

% ==================== 整体标题 ====================
\node[above=0.5cm of ui, font=\Large\bfseries] {教师风格画像分析系统 - 四层架构设计};

\end{tikzpicture}
\end{document}
