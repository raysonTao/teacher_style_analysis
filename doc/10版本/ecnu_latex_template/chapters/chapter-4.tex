\chapter{教师风格画像分析系统设计与实现}


本章在第三章算法研究的基础上,将SHAPE多模态教师风格识别模型转化为可实际部署的教育应用系统。系统以"数据采集—特征提取—风格识别—画像呈现"为主线,构建从课堂录像到教师风格画像的完整处理流程,为教育工作者提供数据驱动的教学风格分析工具。


\section{系统总体设计}


\subsection{系统设计原则}

本系统的设计遵循三项核心原则。

\textbf{模块化与可扩展性}是首要原则。系统采用微服务架构,将视频采集、特征提取、模型推理、画像生成和用户交互等功能解耦为独立模块,各模块可独立部署与升级,互不影响。模型推理层与特征提取层之间通过标准接口通信,使得算法版本迭代无需改动上层应用逻辑。此外,系统预留了扩展接口,可在未来接入眼动追踪、生理信号等新型模态数据,支持功能的渐进式扩展。

\textbf{可解释性与教育适用性}是系统区别于通用机器学习平台的核心价值所在。模型输出不仅包含7类风格的概率分布,还同步提供SHAP特征贡献度与跨模态注意力权重,使每一次风格识别结果都具备可追溯的特征依据。系统将模型输出映射为教育学术语(如"walking频率0.52→巡视互动积极"),并提供典型片段回放功能,帮助教师直观理解系统的判断逻辑。这种设计使系统能够被一线教师和教研人员接受和信任,而非仅作为技术验证工具。

\textbf{高性能与可靠性}是系统工程实现的基本要求。系统采用GPU加速推理,通过NVIDIA TensorRT对模型进行优化,单个10秒语义片段的完整处理时间控制在1.5秒以内。同时,系统设计了三级缓存机制:对已分析视频的模型输出进行24小时缓存,对特征向量进行7天缓存,对视频原始文件进行持久存储,使重复分析的响应时间降至百毫秒级。批处理模式支持对35节课(约35小时)课堂录像进行离线批量分析,满足学校规模化应用需求。


\subsection{系统总体架构}

系统采用五层架构设计(见图~\ref{fig:sys-5layer}),自底向上依次为:数据管理层、特征提取层、模型推理层、应用服务层和用户交互层。

\begin{figure}[htbp]
\centering
\begin{tikzpicture}[
  node distance=0.5cm,
  layer/.style={rectangle, rounded corners=4pt, draw=black!60, fill=#1,
                text width=13.5cm, minimum height=1.05cm,
                align=center, font=\small},
  arrow/.style={-Stealth, thick, draw=black!60},
]
  \node[layer=blue!20]   (L5) {\textbf{Layer 5:用户交互层}\ \ Vue 3 + ECharts 5.4,RESTful API,教师端 / 教研端双视图};
  \node[layer=green!18, below=of L5]  (L4) {\textbf{Layer 4:应用服务层}\ \ Flask 2.3 + Gunicorn,Celery + RabbitMQ,画像生成 / 风格分析服务};
  \node[layer=orange!18, below=of L4] (L3) {\textbf{Layer 3:模型推理层}\ \ SHAPE跨模态注意力融合网络(PyTorch 2.0 + TensorRT 8.5),SHAP解释器};
  \node[layer=purple!15, below=of L3] (L2) {\textbf{Layer 2:特征提取层}\ \ 视频流水线(YOLOv8+DeepSORT+MediaPipe+ST-GCN)\ \textbar\ 音频(Wav2Vec 2.0)\ \textbar\ 文本(Whisper+BERT+H-DAR)};
  \node[layer=gray!20,  below=of L2] (L1) {\textbf{Layer 1:数据管理层}\ \ MinIO对象存储(视频)\ \textbar\ Redis缓存(特征/任务)\ \textbar\ MySQL(元数据)};

  \draw[arrow] (L1.north) -- (L2.south);
  \draw[arrow] (L2.north) -- (L3.south);
  \draw[arrow] (L3.north) -- (L4.south);
  \draw[arrow] (L4.north) -- (L5.south);
\end{tikzpicture}
\caption{教师风格画像分析系统五层架构}
\label{fig:sys-5layer}
\end{figure}

\textbf{数据管理层}(Layer 1)负责原始数据的存储与管理。视频文件通过MinIO对象存储系统保存,支持断点续传和大文件分片上传;特征向量与模型输出缓存于Redis,设置差异化的过期策略;课程信息、教师档案和分析记录等结构化元数据存储于MySQL关系型数据库。

\textbf{特征提取层}(Layer 2)实现三条模态特征提取流水线的并行处理。视频流水线依次经由YOLOv8目标检测、DeepSORT身份追踪、MediaPipe姿态估计和ST-GCN动作识别,完成20维视频特征编码,处理耗时约0.82秒每片段;音频流水线通过Whisper语音识别、Wav2Vec 2.0声学表征提取和情感分类器,生成15维音频特征,耗时约0.37秒;文本流水线经语义分段、H-DAR层次对话行为识别和NLP统计特征提取,生成35维文本特征,耗时约0.15秒。三条流水线并行执行,总特征提取时间约0.82秒(由最慢的视频流水线决定)。

\textbf{模型推理层}(Layer 3)运行SHAPE跨模态注意力融合网络,接收70维联合特征向量,输出7类风格的概率分布及跨模态注意力权重。同层还运行SHAP解释器,对每次预测结果计算70维特征的贡献度评分。该层基于PyTorch 2.0构建,并通过TensorRT 8.5进行推理加速。

\textbf{应用服务层}(Layer 4)基于Flask 2.3框架实现,提供画像生成服务(雷达图、行为柱状图、情绪曲线、关键词云等)和风格分析服务(风格相似度计算、成长曲线追踪、可解释性分析)。服务通过Gunicorn多进程部署,并借助Celery任务队列与RabbitMQ消息中间件实现视频分析任务的异步处理,支持任务优先级调度和失败自动重试(最多3次)。

\textbf{用户交互层}(Layer 5)基于Vue 3框架构建单页面应用,使用ECharts 5.4实现数据可视化,通过RESTful API与应用服务层通信。界面分为教师端(个人风格画像查看、特征分析、风格演变追踪)和教研端(批量分析、跨教师对比、数据导出)两个视图。

在部署架构上,系统支持两种模式:\textbf{单机部署模式}面向校内试点,采用配备NVIDIA RTX 4090 GPU的服务器通过Docker Compose一键启动,可同时处理3路视频的并行分析;\textbf{分布式部署模式}面向区域规模推广,通过Nginx负载均衡、多节点Flask应用服务器和GPU推理服务器的协同,批量处理35节课的耗时可从58分钟压缩至约15分钟。


\section{系统功能模块设计}


\subsection{多模态特征提取流水线}

多模态特征提取模块是系统数据处理的核心环节,负责从课堂录像中提取视频、音频、文本三类模态特征,为后续的风格识别提供标准化的特征输入。

该模块对一节45分钟课堂录像的处理流程如下:首先,系统调用FFmpeg对视频进行格式预处理,将原始录像统一转换为1080p/25fps的MP4格式,并将音频轨提取为16kHz/16bit的WAV文件。接着,系统依据3.2节所述的语义驱动分段策略对音频进行语音识别与意图边界检测,将连续录像切分为若干语义完整的教学片段(平均时长约20秒)。每个片段随后进入三条并行流水线,分别提取视频特征 $F_v \in \mathbb{R}^{20}$、音频特征 $F_a \in \mathbb{R}^{15}$ 和文本特征 $F_t \in \mathbb{R}^{35}$。三路特征合并后形成70维联合表示,送入SHAPE模型进行推理。

特征提取完成后,系统将结果写入Redis缓存,有效期为7天。当同一视频需要重新分析时,系统优先读取缓存特征,将总处理时间从原来的约1小时降至不足1分钟。


\subsection{风格分类推理服务}

风格分类推理服务接收特征提取模块输出的70维特征向量序列,完成从特征到风格标签的映射,是系统的核心计算单元。

推理服务加载预训练的SHAPE模型检查点(最优验证集F1对应的参数),对每个语义片段的70维输入依次完成跨模态注意力计算、BiLSTM时序建模和注意力池化,最终通过Softmax分类层输出7类风格的概率分布 $P(y=k|X) \in [0,1]^7$。对于一节完整课程,服务将所有片段的预测结果按置信度加权聚合,生成课程级风格评分向量(见图~\ref{fig:radar-score}),并以主导风格(最高概率类别)作为该节课的风格标签。

\begin{figure}[htbp]
\centering
\includegraphics[width=0.55\textwidth]{chapters/fig-4/fig-4-2.png}
\caption{课程级风格评分向量七边形雷达图(张三,第3节课)}
\label{fig:radar-score}
\end{figure}

在推理性能方面,服务采用TensorRT对SHAPE模型进行FP16精度量化,单片段推理耗时约0.15秒,整节课(约150个片段)的推理阶段总耗时约22秒。服务通过Flask接口对外暴露,上游的Celery任务工作进程调用该接口,并将结果写入MySQL数据库供前端查询。


\subsection{可解释性分析模块}

可解释性分析模块基于SHAP(SHapley Additive exPlanations)框架,对SHAPE模型的每次预测结果进行特征归因分析,向用户解释"模型为什么做出这一判断"。

SHAP值的计算使用KernelSHAP方法,以测试集全部样本的模型输出均值作为基准值(baseline),通过采样不同特征子集的边际贡献来估算各特征的Shapley值。由于SHAPE模型的输入维度为70维,每次归因计算约需120毫秒,为避免对用户响应时间造成影响,该计算以异步任务的形式执行,用户在查看分析报告时可按需触发。

模块的可视化输出包含三类图表:其一为全局特征重要性条形图(Global Bar Chart),按模态分组展示70维特征的平均绝对SHAP值,帮助用户了解哪类特征对当前教师的风格识别最具判别力;其二为特征分布散点图(Summary Beeswarm),展示各特征取值与SHAP贡献度的对应关系,反映特征对风格判断的方向性影响;其三为单次预测瀑布图(Waterfall Chart),将从基准值到最终预测概率的累积贡献路径完整呈现,使教师能够追踪具体片段的识别依据(如图~\ref{fig:shap-waterfall}所示)。

\begin{figure}[htbp]
\centering
\includegraphics[width=0.82\textwidth]{chapters/fig-4/fig-4-3.png}
\caption{单片段预测"启发引导型"的SHAP特征贡献瀑布图(基准值=0.143)}
\label{fig:shap-waterfall}
\end{figure}


\subsection{风格画像生成模块}

风格画像生成模块将SHAPE模型的分类输出与特征提取结果综合处理,生成多维度的教师风格可视化报告。

\textbf{风格雷达图}(Style Radar Chart)将7类风格的课程级评分映射为七边形雷达图,直观呈现教师风格的分布特征。由于大多数教师的风格并非单一类型,雷达图能有效反映混合风格特征,例如某教师同时具有较高的"理论讲授型"(0.78)和"逻辑推导型"(0.65)评分,表明其擅长以严谨推理支撑知识讲授。

\textbf{行为分布柱状图}(Behavior Histogram)统计一节课中6类教师动作(standing/walking/gesturing/writing/pointing/raise\_hand)的频率与累计持续时间,以双轴柱状图的形式展示,辅助教师了解自身的课堂空间利用与肢体表达模式。

\textbf{语音情绪曲线}(Emotion Curve)以时间为横轴,将45分钟课程中每个语义片段的情感强度(6类情感:neutral/happy/sad/angry/surprise/fear)绘制为折线图,呈现课堂情绪的时序变化趋势,帮助教师识别情感投入较低或情绪波动较大的课堂阶段。

\textbf{关键词云图}(Word Cloud)从Whisper转写的全课文本中提取高频教学术语,经jieba分词和停用词过滤后生成词云,直观展示教师的核心词汇使用模式(如逻辑推导型教师词云中"因为""所以""因此"等逻辑连接词频率显著高于其他类型)。

\textbf{典型片段自动提取}功能从模型预测结果中选取每类风格置信度最高的前3个片段,提供原始视频回放链接,使教师能够直观对照自身被识别为某一风格的具体行为表现,增强画像结果的可信度与教育反馈价值(如图~\ref{fig:typical-seg}所示)。

\begin{figure}[htbp]
\centering
\includegraphics[width=0.90\textwidth]{chapters/fig-4/fig-4-4.png}
\caption{各风格置信度Top-3典型片段展示(3类风格×3片段)}
\label{fig:typical-seg}
\end{figure}


\subsection{风格分析功能}

风格分析功能在风格画像的基础上,提供更深层次的比较与追踪分析,支持教育研究和教师专业发展应用场景。

\textbf{风格相似度评估}通过风格相似度指数(Style Matching Index,SMI)量化教师实际风格与参考风格之间的差异程度:

\[
SMI = 1 - \frac{\displaystyle\sum_{i=1}^{7}\left| S_{\text{target}}^{(i)} - S_{\text{actual}}^{(i)} \right|}{2 \times 7}
\]

其中,$S_{\text{target}}^{(i)}$ 为参考风格的第 $i$ 类评分,$S_{\text{actual}}^{(i)}$ 为当前教师的实际评分,分母 $14$ 为归一化因子。SMI值域为 $[0,1]$,越接近1表示与参考风格越相近。系统内置了四类课型的参考风格模板(理论课、探究课、习题课、复习课),教研人员也可自定义参考风格用于专项研究。需说明的是,SMI仅用于量化风格相似度,不代表教学质量优劣,不同情境下教师需要采用与课型相适配的风格。

\textbf{风格稳定性分析}支持对同一教师跨多节课的风格评分进行时序追踪,计算各风格维度的标准差 $\sigma_k$,反映教师风格的一致性与演变规律。$\sigma_k < 0.10$ 表示高度稳定,$0.10 \leq \sigma_k < 0.20$ 表示较为稳定,$\sigma_k \geq 0.20$ 表示存在明显波动,可能反映教师的风格在不同课型下发生了适应性调整。系统将风格稳定性分析结果以折线图形式呈现(如图~\ref{fig:style-stability}所示),并可叠加显示学期内的课型标注,辅助教师理解风格波动的情境原因。

\begin{figure}[htbp]
\centering
\includegraphics[width=0.90\textwidth]{chapters/fig-4/fig-4-5.png}
\caption{张三跨学期8节课风格稳定性追踪折线图(含线性趋势线)}
\label{fig:style-stability}
\end{figure}


\section{技术栈选型}


\subsection{前端技术}

Vue.js是由尤雨溪主导开发的渐进式JavaScript框架,采用组件化开发模式和响应式数据绑定机制,支持单页面应用(SPA)的构建。Vue 3引入了Composition API,使复杂逻辑的组织和复用更加灵活,同时在性能方面相较Vue 2有显著提升。在本系统中,Vue 3用于构建教师端和教研端的前端界面,各功能页面(画像查看、追踪分析、批量管理等)以独立组件形式开发,通过Vue Router进行路由管理,通过Pinia进行状态管理,实现了界面逻辑的清晰分层。

ECharts是由Apache基金会维护的开源数据可视化库,支持折线图、柱状图、雷达图、散点图、词云图等丰富的图表类型,并提供完善的交互能力(缩放、拖拽、数据筛选等)。本系统选用ECharts 5.4作为可视化引擎,实现了风格雷达图、行为柱状图、情绪折线图、SHAP条形图、成长曲线等全部可视化组件。ECharts的SVG/Canvas双渲染模式使其在不同网络环境和设备分辨率下均能保持良好的显示效果。


\subsection{后端技术}

Flask是Python生态中轻量级的WSGI Web框架,以"微框架"理念著称,核心功能精简但扩展性强。相比Django等重型框架,Flask对PyTorch生态的集成更为自然,适合以机器学习推理为核心的AI应用服务。本系统采用Flask 2.3构建RESTful API服务,通过Blueprint对不同业务模块(分析任务、画像查询、用户管理等)进行路由分组,并配合Gunicorn多进程服务器实现生产环境的并发处理,单机支持50个并发用户的分析请求。

Celery是Python领域主流的分布式任务队列框架,通过消息中间件(本系统采用RabbitMQ)实现任务的异步分发与执行。在本系统中,课堂视频的分析任务(约1小时处理时间)通过Celery以异步方式提交,用户上传视频后立即获得任务ID,前端通过轮询接口查询任务状态,避免了长时间HTTP连接阻塞。Celery工作进程支持优先级队列配置(实时分析任务优先于批量分析任务)和失败自动重试机制(最多3次,指数退避策略),保证了任务执行的可靠性。


\subsection{模型推理技术}

PyTorch是由Meta AI Research开发的开源深度学习框架,以动态计算图和Pythonic的API设计著称,广泛应用于学术研究和工业部署。本系统使用PyTorch 2.0实现SHAPE模型的训练与推理,其`torch.compile()`编译优化功能可在现有代码基础上自动降低推理延迟。

NVIDIA TensorRT是面向NVIDIA GPU的高性能深度学习推理优化库,通过层融合、精度校准和内核自动调优等技术显著提升模型推理速度。在本系统中,训练完成的SHAPE PyTorch模型经由TensorRT 8.5转换为优化的推理引擎,采用FP16混合精度模式,在不显著损失准确率的前提下,推理吞吐量提升约1.8倍,单片段推理时间从原生PyTorch的0.27秒降至0.15秒。


\subsection{数据存储技术}

MySQL是使用最广泛的开源关系型数据库系统,具有成熟的事务支持、索引优化和复制机制。本系统使用MySQL 8.0存储结构化元数据,包括教师信息表(teacher)、课程记录表(lesson)、分析任务表(analysis\_task)和风格结果表(style\_result),通过SQLAlchemy ORM框架实现对象关系映射,简化数据库操作代码。

Redis是基于内存的键值存储系统,以极低的读写延迟(微秒级)和丰富的数据结构支持著称。本系统使用Redis 7.0实现多级缓存:分析任务的状态信息(TTL=24小时)、提取完成的特征向量(TTL=7天)、Celery任务消息队列等均存储于Redis,有效减少了对数据库和GPU资源的重复调用。

MinIO是兼容Amazon S3 API的开源对象存储系统,支持私有化部署,适合在校园网环境下处理教师课堂视频等敏感教育数据。本系统使用MinIO存储原始录像文件和分析结果中间文件(语义片段视频、音频切片),通过预签名URL机制实现前端的安全直连下载,降低了应用服务器的带宽压力。


\subsection{容器化与监控}

Docker是主流的容器化平台,通过Dockerfile将应用程序及其依赖打包为可移植的容器镜像,消除了"开发环境可用、生产环境不可用"的环境差异问题。本系统使用Docker Compose编排多个服务容器(Flask应用、Celery工作进程、MySQL、Redis、MinIO、RabbitMQ等),通过单条命令实现整套系统的一键启动与停止,极大降低了学校IT人员的运维门槛。

Prometheus与Grafana是云原生监控的经典组合。Prometheus负责从各服务容器中采集运行时指标(CPU/GPU利用率、内存占用、任务队列长度、推理延迟等),Grafana将这些指标以仪表盘形式可视化展示,支持阈值告警(如GPU利用率持续超过90\%时触发告警)。本系统部署了完整的监控栈,为系统管理员提供实时的运行状态视图,辅助容量规划和故障排查。


\section{界面功能描述}

系统前端界面共包含五个主要页面,以下分别描述各页面的布局与核心功能。


\subsection{视频上传与任务管理页面}

视频上传页面(如图~\ref{fig:ui-upload}所示)是用户使用系统的入口。页面左侧提供课程信息填写区域,用户在此输入教师姓名、课程名称、授课日期和课型(理论课/习题课/探究课/复习课)等基本信息。页面中央为拖拽式文件上传区,支持MP4、MOV、AVI等主流视频格式,最大支持单文件8GB的断点续传上传。上传进度以百分比进度条实时显示,上传完成后系统自动创建分析任务并返回任务编号。

任务管理页面(如图~\ref{fig:ui-task}所示)以列表形式展示用户提交的全部分析任务,每条记录显示任务编号、课程信息、提交时间、当前状态(排队中/特征提取中/模型推理中/画像生成中/已完成/失败)和进度百分比。用户可在此页面查看实时进度、取消排队中的任务或重新提交失败任务。对于已完成的任务,点击"查看报告"按钮可跳转至对应的风格画像页面。

\begin{figure}[htbp]
\centering
\includegraphics[width=0.90\textwidth]{chapters/fig-4/fig-4-6.png}
\caption{视频上传页面界面原型}
\label{fig:ui-upload}
\end{figure}

\begin{figure}[htbp]
\centering
\includegraphics[width=0.90\textwidth]{chapters/fig-4/fig-4-7.png}
\caption{分析任务管理页面界面原型}
\label{fig:ui-task}
\end{figure}


\subsection{风格画像综合展示页面}

风格画像页面(如图~\ref{fig:ui-profile}所示)是系统的核心展示界面,以"一图概览、四维详情"的布局呈现分析结果。

页面顶部为课程基本信息栏,显示教师姓名、课程名称、主导风格标签(如"启发引导型(置信度87.3\%)")及分析完成时间。页面主体分为左右两区:左侧占60\%宽度,展示风格雷达图(7类风格评分的七边形可视化),鼠标悬停于各顶点时弹出该风格的详细说明和代表性特征指标;右侧占40\%宽度,展示行为分布柱状图,以双轴形式并列显示6类动作的频率(\%)和持续时长(分钟),支持点击动作类别定位至时序详情。

页面下半部分分为两个Tab面板:第一个Tab展示语音情绪曲线,以折线图呈现课程全程的情感强度时序变化,用户可拖拽时间轴缩放查看;第二个Tab展示关键词云图,词语大小与出现频率正相关,点击词语可在转写文本中定位该词语的出现位置。

\begin{figure}[htbp]
\centering
\includegraphics[width=0.90\textwidth]{chapters/fig-4/fig-4-8.png}
\caption{风格画像综合展示页面界面原型}
\label{fig:ui-profile}
\end{figure}


\subsection{可解释性与特征详情页面}

可解释性页面(如图~\ref{fig:ui-explain}所示)为有进一步分析需求的用户提供特征级的解释工具,面向教研人员和有数据分析能力的教师。

页面顶部展示模态贡献度饼图,直观呈现视觉、音频、文本三种模态在本次预测中的权重占比,数值来源于跨模态注意力权重 $\alpha_{i \to j}$ 的聚合计算(详见第3.5节)。页面主体展示SHAP全局特征重要性条形图,按绝对SHAP值降序列出Top-20特征,不同模态的特征以不同颜色区分(视频特征为蓝色、音频特征为橙色、文本特征为绿色)。将鼠标悬停于条形上,可查看该特征的具体取值及其对预测结果的方向性影响(正向/负向贡献)。

页面下方提供典型片段回放区域,按风格类别分组展示置信度最高的3个视频片段缩略图,点击可在线播放对应的课堂视频片段,使教师能够将系统的量化描述与自身实际行为直接对照。

\begin{figure}[htbp]
\centering
\includegraphics[width=0.90\textwidth]{chapters/fig-4/fig-4-9.png}
\caption{可解释性与特征详情页面界面原型}
\label{fig:ui-explain}
\end{figure}


\subsection{风格演变追踪页面}

风格演变追踪页面(如图~\ref{fig:ui-trend}所示)面向已积累多节课分析记录的教师,提供跨时间段的风格变化分析。

页面顶部为课程筛选区,用户可通过日期范围选择器和课型筛选器确定分析区间,支持对最近一个月、一学期或自定义时间段的课程记录进行追踪。页面主体展示成长曲线图,以折线图呈现各风格维度的评分随时间的变化趋势,同时叠加显示线性回归拟合线,直观反映风格演变的整体方向。用户可通过图例选择显示或隐藏特定风格维度的曲线,避免多线重叠影响可读性。

页面右侧展示风格稳定性分析结果,以热力图形式呈现7类风格在选定时间段内的标准差分布,稳定性较高的维度以深色表示,波动较大的维度以浅色表示,辅助教师识别自身风格的稳定特征与动态调整空间。

\begin{figure}[htbp]
\centering
\includegraphics[width=0.90\textwidth]{chapters/fig-4/fig-4-10.png}
\caption{风格演变追踪页面界面原型}
\label{fig:ui-trend}
\end{figure}


\subsection{批量分析与教研对比页面}

批量分析页面(如图~\ref{fig:ui-batch}所示)面向教研人员,支持对多名教师或多节课程的批量处理与横向对比。

用户在此页面可选择多个已上传的视频任务或批量导入视频文件列表,提交批量分析任务后,系统将任务拆分分发至多个Celery工作进程并行执行,处理35节课的预计耗时约58分钟。批量分析完成后,页面展示多教师风格分布对比雷达图(多组数据叠加在同一雷达图中)和各维度的均值与方差统计,辅助教研人员发现群体共性规律或个体差异特征。结果支持导出为Excel报表,包含每位教师的完整风格评分和主要特征指标。

\begin{figure}[htbp]
\centering
\includegraphics[width=0.90\textwidth]{chapters/fig-4/fig-4-11.png}
\caption{批量分析与教研对比页面界面原型(教研管理员视图)}
\label{fig:ui-batch}
\end{figure}


\section{系统测试与试运行}


\subsection{测试环境}

系统测试在单机部署模式下进行,软硬件环境配置如表4-1所示。

\begin{table}[htbp]
\centering
\caption{测试环境配置}
\label{tab:test-env}
\begin{tabular}{|l|l|l|}
\hline
类别 & 配置项 & 具体配置 \\
\hline
硬件 & CPU & Intel Core i9-13900K(24核) \\
\hline
硬件 & GPU & NVIDIA RTX 3090(24GB VRAM) \\
\hline
硬件 & 内存 & 64GB DDR5 \\
\hline
硬件 & 存储 & 2TB NVMe SSD \\
\hline
软件 & 操作系统 & Ubuntu 22.04 LTS \\
\hline
软件 & 深度学习框架 & PyTorch 2.0.1 + CUDA 11.8 \\
\hline
软件 & Web框架 & Flask 2.3.2 + Gunicorn 21.2 \\
\hline
软件 & 数据库 & MySQL 8.0.33 + Redis 7.0.12 \\
\hline
软件 & 容器 & Docker 24.0 + Docker Compose 2.20 \\
\hline
网络 & 带宽 & 千兆以太网(内网) \\
\hline
\end{tabular}
\end{table}


\subsection{功能性测试}

功能性测试依据系统需求规格,覆盖视频上传、特征提取、风格识别、画像生成和用户交互等核心功能模块。测试采用黑盒方法,依据测试用例的预期输出验证实际输出的正确性。主要测试用例及结果如表4-2所示。

\begin{table}[htbp]
\centering
\caption{功能性测试用例}
\label{tab:func-test}
\resizebox{\linewidth}{!}{%
\begin{tabular}{|l|l|l|l|l|}
\hline
功能模块 & 测试项 & 输入 & 预期结果 & 测试结果 \\
\hline
视频上传 & 正常上传MP4文件 & 1.2GB,45分钟,1080p & 上传成功,返回任务ID & 通过 \\
\hline
视频上传 & 超大文件上传(>2GB) & 4.5GB视频文件 & 断点续传,分片上传成功 & 通过 \\
\hline
视频上传 & 不支持格式上传 & .wmv格式文件 & 返回格式错误提示 & 通过 \\
\hline
特征提取 & 单教师场景提取 & 含单一教师的课堂录像 & 成功提取70维特征 & 通过 \\
\hline
特征提取 & 多人场景教师识别 & 含多人的课堂录像 & 正确定位并追踪主讲教师 & 通过 \\
\hline
特征提取 & 噪声环境音频处理 & SNR=10dB低信噪比录音 & 成功提取音频特征 & 通过 \\
\hline
风格识别 & 标准样本分类 & 测试集已标注样本 & 风格标签与标注一致 & 通过 \\
\hline
风格识别 & 混合风格识别 & 多风格混合的课堂片段 & 输出合理的概率分布 & 通过 \\
\hline
画像生成 & 雷达图生成 & 7维风格评分向量 & 正确渲染交互式雷达图 & 通过 \\
\hline
画像生成 & 情绪曲线生成 & 时序情感标签序列 & 折线图按时间顺序渲染 & 通过 \\
\hline
画像生成 & 关键词云图生成 & 课程全文转写文本 & 生成词频正确的云图 & 通过 \\
\hline
追踪分析 & 风格稳定性计算 & 同一教师10节课记录 & 标准差计算正确 & 通过 \\
\hline
追踪分析 & SMI相似度计算 & 当前评分与参考模板 & SMI值在[0,1]范围内 & 通过 \\
\hline
SHAP分析 & 特征归因计算 & 单次预测特征向量 & 返回70维SHAP值 & 通过 \\
\hline
典型片段 & 片段自动提取 & 片段置信度列表 & 每类风格提取Top-3片段 & 通过 \\
\hline
批量分析 & 多任务并行提交 & 5节课同时提交分析 & 全部任务成功完成 & 通过 \\
\hline
任务管理 & 任务状态查询 & 进行中的任务ID & 返回当前处理进度 & 通过 \\
\hline
任务管理 & 失败任务重试 & 模拟服务异常后重试 & 自动重试后成功完成 & 通过 \\
\hline
\end{tabular}}
\end{table}

全部18项功能测试用例均通过验证,系统核心功能运行正常。


\subsection{非功能性测试}

\textbf{(一)性能测试}

性能测试重点评估系统在不同负载条件下的处理能力与响应时间,结果如表4-3所示。

\begin{table}[htbp]
\centering
\caption{性能测试结果}
\label{tab:perf-test}
\begin{tabular}{|l|l|l|l|}
\hline
测试场景 & 测试指标 & 目标值 & 实测值 \\
\hline
单片段处理(10秒) & 端到端处理时间 & ≤1.5秒 & 1.34秒 \\
\hline
整课分析(45分钟) & 总处理时间 & ≤75分钟 & 约62分钟 \\
\hline
缓存命中分析 & 重复分析响应时间 & ≤1分钟 & 约45秒 \\
\hline
并发用户(50人) & 系统响应时间 & ≤3秒 & 2.1秒 \\
\hline
批量分析(35节课) & 总处理时间 & ≤90分钟 & 58分钟 \\
\hline
SHAP归因计算 & 单样本耗时 & ≤200毫秒 & 120毫秒 \\
\hline
\end{tabular}
\end{table}

\begin{sloppypar}
单片段1.34秒的端到端处理时间中,视频流水线(YOLOv8+DeepSORT+MediaPipe+ST-GCN)耗时0.82秒,音频流水线耗时0.37秒,SHAPE推理耗时0.15秒,三条流水线并行执行,总时间由最慢的视频流水线决定。各项性能指标均达到设计目标。
\end{sloppypar}

\textbf{(二)安全性测试}

系统进行了以下安全性验证:在接口访问控制方面,所有API接口均实现JWT(JSON Web Token)身份认证,未授权请求返回401状态码,测试中连续尝试50次未授权访问均被正确拒绝;在文件上传安全方面,系统对上传文件进行MIME类型校验和文件头魔数验证,测试中上传的伪造扩展名可执行文件(将.exe改名为.mp4)均被正确识别并拒绝;在SQL注入防护方面,系统通过SQLAlchemy参数化查询处理所有数据库操作,测试中构造的SQL注入载荷均未引发异常执行;在视频数据安全方面,MinIO存储桶配置为私有访问策略,视频文件通过有效期为15分钟的预签名URL按需提供,有效防止敏感录像的未授权访问。


\subsection{系统试运行}

为验证系统在真实教育场景中的可用性,本研究对自建数据集的55节课录像(涵盖7位教师,总时长约41小时)进行了系统试运行,评估系统在完整使用流程中的稳定性与实用效果。

试运行期间,系统累计处理视频约221GB,生成分析报告55份,完成SHAP归因计算55次,累计运行时间超过60小时,未发生系统崩溃或数据丢失事件。对于因网络波动导致上传中断的3次事件,断点续传功能均成功恢复,用户体验未受明显影响。任务处理阶段发生2次Celery工作进程异常退出,自动重试机制在3分钟内完成了任务的恢复执行,用户端无需手动干预。

在分析结果的可用性方面,参与试运行的7位教师在查看自己的风格画像报告后,均能通过雷达图和典型片段回放对系统的识别结果形成合理认知。其中4位教师表示风格雷达图与其自我认知"基本一致",2位教师表示"部分一致",1位教师对某一维度的识别结果存有疑问,经查阅对应的SHAP特征详情后,认为该判断"有一定道理但与自己预期不同"。整体来看,系统的可解释性设计有效降低了教师对结果的质疑程度,多维度的可视化输出也得到了教师的积极反馈。


\section{本章小结}

本章在第三章算法研究成果的基础上,完成了教师风格画像分析系统的设计与实现,将SHAPE多模态教师风格识别模型转化为面向教育场景的实用系统。

在\textbf{系统设计}方面,系统遵循模块化、可解释性和高性能三项核心原则,采用五层架构(数据管理层、特征提取层、模型推理层、应用服务层、用户交互层)组织各功能模块,通过微服务设计实现各层的独立部署与弹性扩展。系统支持单机部署(校内试点)和分布式部署(区域推广)两种模式,适应不同规模的应用场景。

在\textbf{技术实现}方面,前端采用Vue 3 + ECharts 5.4实现响应式交互界面与多维度可视化;后端采用Flask + Celery + RabbitMQ实现RESTful API服务与异步任务调度;模型推理采用PyTorch + TensorRT实现GPU加速,单片段处理时间1.34秒;数据存储采用MySQL + Redis + MinIO的分层架构,通过三级缓存将重复分析的响应时间降至45秒以内;系统通过Docker Compose实现容器化一键部署,降低运维门槛。

在\textbf{功能实现}方面,系统提供多模态特征提取、风格分类推理、可解释性分析(SHAP)、风格画像可视化(雷达图、行为柱状图、情绪曲线、关键词云、典型片段)、风格相似度评估(SMI)、风格稳定性追踪六大核心功能,覆盖从视频上传到风格报告生成的完整使用流程。

在\textbf{测试验证}方面,系统通过了18项功能测试(全部通过)、6项性能测试(各项指标均达到设计目标)和多项安全性测试,并在55节课录像的试运行中保持稳定运行,未发生系统级故障。试运行中参与教师的反馈表明,系统的可解释性设计有效提升了画像结果的可信度与用户接受度,验证了将SHAPE算法工程化落地的可行性。

