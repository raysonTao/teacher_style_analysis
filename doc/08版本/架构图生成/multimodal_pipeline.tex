
\documentclass[border=8pt, multi, tikz]{standalone}
\usepackage{tikz}
\usepackage{xcolor}
\usepackage{amsmath}
\usepackage{amssymb}
\usepackage{xeCJK}
\setCJKmainfont{Noto Sans CJK SC}

\usetikzlibrary{positioning, arrows.meta, shapes.geometric, calc, fit}

% 定义颜色
\definecolor{video}{RGB}{255,127,127}
\definecolor{audio}{RGB}{127,255,127}
\definecolor{text}{RGB}{127,127,255}

\begin{document}
\begin{tikzpicture}[
    node distance=0.6cm and 1cm,
    process/.style={rectangle, draw, thick, minimum width=3cm, minimum height=0.8cm, align=center, font=\footnotesize, rounded corners=2pt},
    data/.style={rectangle, draw, dashed, minimum width=2.5cm, minimum height=0.6cm, align=center, font=\scriptsize},
    arrow/.style={-Stealth, thick},
    label/.style={font=\tiny, align=center}
]

% ==================== 视频流水线 ====================
\node[data, fill=video!20] (video_in) at (0,0) {视频片段\\10s@25fps};
\node[process, fill=video!30, below=of video_in] (extract) {ExtractFrames\\250帧};
\node[process, fill=video!40, below=of extract] (yolo) {YOLOv8-Batch\\人体检测};
\node[process, fill=video!50, below=of yolo] (deepsort) {DeepSORT\\教师追踪};
\node[process, fill=video!60, below=of deepsort] (mediapipe) {MediaPipe\\姿态估计};
\node[process, fill=video!70, below=of mediapipe] (stgcn) {ST-GCN\\动作识别};
\node[data, fill=video!80, below=of stgcn] (video_out) {$F_v \in \mathbb{R}^{20}$};

\draw[arrow] (video_in) -- (extract);
\draw[arrow] (extract) -- node[right, label] {0.05s} (yolo);
\draw[arrow] (yolo) -- node[right, label] {0.18s} (deepsort);
\draw[arrow] (deepsort) -- node[right, label] {0.12s} (mediapipe);
\draw[arrow] (mediapipe) -- node[right, label] {0.25s} (stgcn);
\draw[arrow] (stgcn) -- node[right, label] {0.18s} (video_out);

\node[above=0.3cm of video_in, font=\bfseries] {视频流水线 (0.82s)};

% ==================== 音频流水线 ====================
\node[data, fill=audio!20, right=4cm of video_in] (audio_in) {音频片段\\10s@16kHz};
\node[process, fill=audio!30, below=of audio_in] (load) {LoadAudio\\160k采样点};
\node[process, fill=audio!40, below=of load] (whisper) {Whisper\\语音转写};
\node[process, fill=audio!50, below=of whisper] (wav2vec) {Wav2Vec2\\声学嵌入};
\node[process, fill=audio!60, below=of wav2vec] (emotion) {Emotion\\情感分类};
\node[data, fill=audio!70, below=of emotion] (audio_out) {$F_a \in \mathbb{R}^{15}$};

\draw[arrow] (audio_in) -- (load);
\draw[arrow] (load) -- node[right, label] {0.05s} (whisper);
\draw[arrow] (whisper) -- node[right, label] {0.15s} (wav2vec);
\draw[arrow] (wav2vec) -- node[right, label] {0.08s} (emotion);
\draw[arrow] (emotion) -- node[right, label] {0.07s} (audio_out);

\node[above=0.3cm of audio_in, font=\bfseries] {音频流水线 (0.37s)};

% ==================== 文本流水线 ====================
\node[data, fill=text!20, right=4cm of audio_in] (text_in) {转写文本\\(await)};
\node[process, fill=text!30, below=of text_in] (bert) {BERT\\语义编码};
\node[process, fill=text!40, below=of bert] (hdar) {H-DAR\\意图识别};
\node[process, fill=text!50, below=of hdar] (nlp) {ComputeNLP\\统计特征};
\node[data, fill=text!60, below=of nlp] (text_out) {$F_t \in \mathbb{R}^{35}$};

\draw[arrow] (text_in) -- (bert);
\draw[arrow] (bert) -- node[right, label] {0.06s} (hdar);
\draw[arrow] (hdar) -- node[right, label] {0.04s} (nlp);
\draw[arrow] (nlp) -- node[right, label] {0.05s} (text_out);

% 依赖关系
\draw[arrow, dashed, color=gray] (whisper.east) -- ++(0.5, 0) |- (text_in.west);

\node[above=0.3cm of text_in, font=\bfseries] {文本流水线 (0.15s)};

% ==================== 并行合并 ====================
\node[process, fill=gray!30, below=3cm of audio_out, minimum width=8cm, minimum height=1cm] (merge) {
    \textbf{特征合并} \\
    $F = \{F_v, F_a, F_t\} \in \mathbb{R}^{70}$
};

\draw[arrow] (video_out) -| (merge);
\draw[arrow] (audio_out) -- (merge);
\draw[arrow] (text_out) -| (merge);

% ==================== 并行标注 ====================
\node[draw, dashed, thick, fit=(video_in)(video_out)(audio_in)(audio_out)(text_in)(text_out),
      label={[font=\bfseries]above:Pipeline 并行处理}] {};

% ==================== 整体标题 ====================
\node[above=1cm of video_in, font=\Large\bfseries] {多模态特征提取流程图 (Algorithm 1)};

\end{tikzpicture}
\end{document}
