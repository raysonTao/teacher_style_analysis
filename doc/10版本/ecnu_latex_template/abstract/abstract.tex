\chapter*{\xiaosan\heiti{摘\quad 要}}
\addcontentsline{toc}{chapter}{摘要}

在教育数字化转型的浪潮中,海量课堂录像数据亟待被有效利用以赋能教学。教师教学风格是影响课堂质量的关键因素,但传统评价方法主观性强、反馈滞后,难以满足智慧教育环境下对客观、实时、可量化课堂反馈的需求。为此,本研究设计并实现了一个基于多模态深度学习的教师教学风格画像分析系统,旨在提供客观、精细、可解释的智能风格画像分析。

现有课堂分析技术存在单模态视频或音频难以全面刻画教学风格和风格识别无法提供决策依据和特征贡献度等问题。针对上述挑战,本研究提出了\textbf{SHAPE(Semantic Hierarchical Attention Profiling Engine,语义层次化注意力画像引擎)},通过语义驱动分段、层次化教学意图识别和跨模态注意力机制实现特征的自适应融合与风格的精准画像。具体包括:数据分段策略优化方面,提出语义驱动的话语分段策略,通过依存句法分析和话语边界检测,保持教学话语的语义完整性,使教学意图识别准确率大幅度提升;音频模态方面,不仅将音频用于语音情绪识别,在课堂场景下进行微调,使用自动语音识别(ASR)技术将音频转化为文本模态,为意图识别打下基础;文本模态方面,引入基于BERT的层次化细粒度对话行为识别(Hierarchical Dialogue Act Recognition),采用两层分类架构(粗分类4类+细分类10类),将单层分类扩展为双层10类细粒度分类,更有效地捕捉不同教学风格的特征性语言模式;视觉模态方面,使用身份识别算法实现稳定的教师身份追踪,并采用时空图卷积网络对骨骼序列进行时序建模,相比单帧动作识别准确率较大提升;智能融合与解释方面,设计的SHAPE通过跨模态注意力机制自适应地融合视觉、音频、文本特征,并结合注意力权重与SHAP可解释性分析,提升模型决策依据的可追溯性。

在自建的教师风格数据集(209个样本,7类风格)上,SHAPE在风格识别任务中取得了\textbf{92.5\%}的准确率,显著优于单一模态方法和简单融合方法。消融实验进一步证实,语义驱动分段策略使风格识别准确率提升\textbf{7.6个百分点}(McNemar $\chi^2=4.00$,$p<0.05$),验证了这些改进的有效性。

同时构建出一套教师课堂画像系统,将上述算法的结果进行可视化。本系统能够生成直观、可追溯的教师风格画像(风格雷达图、模态贡献度分析、典型片段回放),为教师风格认知和教学研究提供了科学、客观、精细化的数据支撑。

\vspace{0.5cm}
\sihao{\heiti{关键词:}}\xiaosi{教师教学风格;多模态学习分析;跨模态注意力;深度学习}
